\subsection{Zadanie 1 -- Najmniejszy zbiór dominujący}
\paragraph{Treść:}
\textit{Zbiór dominujący} w grafie $G$ to podzbiór wierzchołków 
$D \subset V(G)$ taki, że każdy wierzchołek grafu $G$
należy do $D$ lub ma przynajmniej jednego sąsiada w $D$.

Zaprojektuj jak najwydajniejszy algorytm, który znajdzie rozmiar najmniejszego zbioru dominującego w zadanym
grafie o $n$ wierzchołkach i maksymalnym stopniu nie większym niż 3. Oszacuj złożoność tego algorytmu.

\paragraph{Rozwiązanie:}
Aby rozwiązać to zadanie zastosujemy algorytm z nawrotami.

Wprowadźmy oznaczenia, które zostaną wykorzystane w algorytmie.
Niech $D_{\min}$ to najmniejszy zbiór dominujący jaki udało 
nam się dotychczas znaleźć.
Niech $D$ to aktualny stan zbioru dominującego, oraz
niech $X$ to zbiór wierzchołków, które na pewno nie znajdą się 
w zbiorze dominującym $D$ na następnych poziomach rekurencji.
Warunkiem stopu będzie sytuacja, w której $|D| + |X| = |V(G)|$.

Możemy wprowadzić optymalizację polegająca na tym, że 
przestaniemy badać daną gałąź wykonań rekurencyjnych,
w przypadku, kiedy $|D| < |D_{\min}|$. Optymalizacji tej nie będziemy
stosowali w przypadku kiedy $D_{\min}$ nie zostało jeszcze znalezione,
w tym celu skorzystamy z faktu, że jeśli graf $G$, jest 
taki, że $V(G) \not = \emptyset$,
to każdy zbiór dominujący dla tego grafu nie jest pusty.

W poniższym algorytmie, przyjmujemy, że $D_{\min}$ to globalny zbiór
wierzchołków, który początkowo jest pusty.
\begin{algorithm}[H]
	\caption{Algorytm znajdowania najmniejszego zbioru dominującego}
	\begin{algorithmic}[1]
		\Procedure{DominatingSet}{$G$:graf, $D$:zbiór, $X$:zbiór}
		\If{$|D| > |D_{min}| \land |D_{min}| \not = \emptyset$}
		\State \Return
		\State Niech $Y$ to zbiór wierzchołków niezdominowanych, tzn. takich, które nie mają żadnego sąsiada w zbiorze $D$ (żaden
		wierzchołek z $Y$ nie należy do $X$ ani $D$)
		\EndIf		
		\If{$Y = \emptyset$}
		\If{Każdy wierzchołek z $X$ jest zdominowany}
		\State $D_{\min} = D$
		\EndIf
		\State \Return
		\EndIf
		\State Niech $u$ to dowolny wierzchołek z $Y$
		\State DominatingSet($G$, $D \cup \{u\}$, $X$, $D_{\min}$)
		\If{$|N(u)| \geq 1$}
		\State Niech $v_1 \in N(X)$
		\If{$v_1 \not \in D \land v_1 \not \in X$}
		\State DominatingSet($G$, $D \cup \{v_1\}$, $X \cup \{u\}$, $D_{\min}$)
		\EndIf
		\EndIf
		\If{$|N(u)| \geq 2$}
		\State Niech $v_2 \in N(X) \setminus \{v_1\}$
		\If{$v_2 \not \in D \land v_2 \not \in X$}
		\State DominatingSet($G$, $D \cup \{v_2\}$, $X \cup \{u, v_1\}$, $D_{\min}$)
		\EndIf
		\EndIf
		\If{$|N(u)| = 3$}
		\State Niech $v_3 \in N(X) \setminus \{v_1, v_2\}$
		\If{$v_3 \not \in D \land v_3 \not \in X$}
		\State DominatingSet($G$, $D \cup \{v_3\}$, $X \cup \{u, v_1, v_2\}$, $D_{\min}$)
		\EndIf
		\EndIf
		\EndProcedure		
	\end{algorithmic}
\end{algorithm}

Aby oszacować złożoność powyższego algorytmu skorzystamy z 
twierdzenia o rozwiązywaniu rekurencji. Przez $f(n)$
oznaczmy pesymistyczny czas działania funkcji \textsc{DominatingSet},
wtedy prawdą jest, że
\[f(n) \leq f(n-1) + f(n-1) + f(n-2) + f(n-3) + f(n-4) + O(n),\]
gdzie $O(n)$ to złożoność znajdowania zbioru $Y$. 
Szukamy pierwiastka poniższego równania
\[x^4 = x^3 + x^2 + x + 1,\]
po rozwiązaniu powyższego równania, otrzymamy największy
pierwiastek $x \approx 1.9276$. Zatem na mocy twierdzenia
o rozwiązywaniu rekurencji złożoność algorytmu wynosi $O^(n1.9276^n) = O^*(1.9276^n)$.
