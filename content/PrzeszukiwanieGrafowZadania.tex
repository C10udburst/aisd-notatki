\subsection{Struktura danych reprezentująca graf skierowany (2.1)}

\paragraph{Treść.} Zaprojektuj strukturę danych reprezentującą graf skierowany o $n$ wierzchołkach i $m$ krawędziach, z
następującymi operacjami:
\begin{itemize}
	\item dodanie krawędzi,
	\item usunięcie krawędzi,
	\item sprawdzenie istnienia krawędzi,
	\item przejrzenie wszystkich krawędzi wychodzących z zadanego wierzchołka $v$,
\end{itemize}
która spełnia następujące wymagania:
\begin{enumerate}[label=\alph*)]
	\item Operacje na pojedynczej krawędzi w czasie $O(\log n)$, przejrzenie krawędzi wychodzących z $v$ w czasie $O(\deg^+(v))$\daggerfootnote{$\deg^+(v)$ oznacza liczbę krawędzi wychodzących z wierzchołka $v$.}, 
	złożoność pamięciowa $O(m)$.
	\item Dodawanie krawędzi w czasie $O(1)$, przejrzenie krawędzi wychodzących z $v$ w czasie $O(\deg^+(v))$, złożoność
	pamięciowa $O(m)$.
	\item Operacje na pojedynczej krawędzi w czasie $O(1)$, przejrzenie krawędzi wychodzących z $v$ w czasie $O(\deg^+(v))$.
	\item Operacje na pojedynczej krawędzi w oczekiwanym czasie $O(1)$, przejrzenie krawędzi wychodzących z $v$ w czasie
	$O(\deg^+(v))$, złożoność pamięciowa $O(m)$.
\end{enumerate}

\paragraph{Rozwiązanie.}
\begin{enumerate}[label=\alph*)]
	\item Reprezentacja listowa, w której 
	listy zastępujemy drzewami zrównoważonymi (CC, AVL).
	\item Lista nieuporządkowana dla każdego wierzchołka
	\item Reprezentacja macierzowa, która przechowuje
	referencję na node w liście incydencji (mamy listę incydencji zawierającą wartości i tablice referencji)
	\item Tablica haszująca o $m$ elementach (haszowanie łańcuchowe)
\end{enumerate}

\subsection{Nierekurencyjne przeszukiwanie w głąb -- DFS (2.2)}
\label{zad:dfsnr}
\paragraph{Treść.} Zaproponuj nierekurencyjną implementację
przeszukiwania grafu w głąb.
\paragraph{Rozwiązanie.}
Poniższy algorytm działa zarówno dla grafów skierowanych
jak i prostych.

\begin{algorithm}[H]
	\caption{Przeszukiwanie grafu w głąb}\label{Zadanie22a}
	\begin{algorithmic}[1]
		\Procedure{DFSNR}{$G$: graf, $v$: wierzchołek startowy}
		\State Zainicjalizuj stos (\textit{last in, first out}) $S$
		\State Zainicjalizuj tablicę booli $T$ o $|V(G)|$ elementach wartością 
		\textit{false}
		\State $S$.Push(v)
		\State $T[v] \gets$ \false
		\While {$S$.IsEmpty() = \false}
		\State $x \gets$ $S$.Pop()
		\While {$y \in N(x)$}
		\If {$T[y]$ = \false}
		\State $S$.Push(y)
		\State $T[y] \gets$ \true 
		\EndIf
		\EndWhile
		\EndWhile
		\EndProcedure
	\end{algorithmic}
\end{algorithm}

\paragraph{Wersja rekurencyjna.} Zakładamy, że dysponujemy globalną
tablicą booli $T$ o $|V(G)|$ elementach zainicjowaną wartością 
\textsf{false}.

\begin{algorithm}[H]
	\caption{Algorytm przeszukiwania DFS -- wersja rekurencyjna}\label{Zadanie22b}
	\begin{algorithmic}[1]
		\Procedure{DFS}{$G$: graf, $v$: wierzchołek startowy}
		\While{$x \in N(v)$}
		\If {$T[x] = $ \textit{false} }
		\State $T[x] \gets$ \textit{true}
		\State \textsc{DFS}($G$, $x$)
		\EndIf
		\EndWhile
		\EndProcedure
	\end{algorithmic}
	\label{alg:dfs}
\end{algorithm}

\paragraph{Oszacowanie złożoności.} Przyjmijmy bez straty ogólności, że graf wejściowy
$G$ jest spójny. W takiej sytuacji algorytm DFS odwiedzi dokładnie $n$ wierzchołków. Ponadto
przeszukujemy nieodwiedzonych sąsiadów badanego wierzchołka, co sprawia, że w całym
wykonaniu algorytmu \ref{alg:dfs}, linijka nr 2 wykona się dokładnie $m$ razy.

Z powyższych rozważań, wynika, że złożoność algorytmu DFS wynosi $O(n + m)$. Zauważmy, że
w przypadku kiedy graf wejściowy $G$ ma więcej niż $m$ krawędzi, możemy przyjąć złożoność $O(m)$.

\subsection{Implementacja przeszukiwania wszerz -- BFS (2.3)}
\label{zad:bfs}
\paragraph{Treść.} Zaproponuj implementację
przeszukiwania grafu wszerz.
\paragraph{Rozwiązanie.}
Poniższy algorytm działa zarówno dla grafów skierowanych
jak i prostych.

\begin{algorithm}[H]
	\caption{Przeszukiwanie grafu wszerz}\label{Zadanie23}
	\begin{algorithmic}[1]
		\Procedure{BFS}{$G$: graf, $v$: wierzchołek startowy}
		\State Zainicjalizuj kolejkę (\textit{first in, first out}) $Q$
		\State Zainicjalizuj tablicę booli $T$ o $|V(G)|$ elementach wartością 
		\false
		\State $Q$.PushBack(v)
		\State $T[v] \gets$ \false
		\While {$Q$.IsEmpty() $=$ \textit{false}}
		\State $x \gets$ $Q$.PopFront()
		\While {$y \in N(x)$}
		\If {$T[y] =$  \false}
		\State $Q$.PushBack(y)
		\State $T[y] \gets$ \true
		\EndIf
		\EndWhile
		\EndWhile
		\EndProcedure
	\end{algorithmic}
\end{algorithm}

\textbf{Oszacowanie złożoności:} Analogiczne do algorytmu DFS (algorytm \ref{alg:dfs}).


\subsection{Zadanie 4 -- Szukanie liczby składowych spójności grafu}
\paragraph{Treść.} Skonstruuj algorytm znajdujący liczbę składowych 
spójności zadanego grafu w czasie $O(m)$.

\paragraph{Rozwiązanie.} Stosujemy \hyperref[Zadanie23]{BFS} lub \hyperref[Zadanie22a]{DFS} z dowolnego wierzchołka i zapisujemy odwiedzone wierzchołki (dla uproszczenia zakładamy, że algorytmy te jako dodatkowy argument przyjmują tablicę $T$ typu \textit{bool}). Jeżeli po zakończeniu wyszukiwania istnieje jakiś nieodwiedzony wierzchołek $x \in V(G)$, to ponownie wykonujemy BFS lub DFS, tym razem z wierzchołka $x$. Powtarzamy tę czynność tak długo, aż wszystkie wierzchołki będą odwiedzone. 



Liczba wywołań BFS lub DFS stanowi liczbę składowych w grafie $G$.

\begin{algorithm}[H]
	\caption{Znajdowanie liczby składowych}\label{Zadanie24}
	\begin{algorithmic}[1]
		\Procedure{ComponentCount($G$: graf)}{}
		\State Zainicjalizuj tablicę booli $T$ o $|V(G)|$ elementach wartością \false
		\State $count\gets0$
		\For{$v=0;\;\;v<|V(G)|;\;\;v\pp$}
			%\If{$T[i]$}\State\textbf{continue}\EndIf
			\State \algorithmicif\ $T[v]$\ \algorithmicthen\ \textbf{continue} % wersja inline, wolę ją, ale u do u
			\State $count\pp$
			\State BFS($G$, $v$, $T$)\quad albo\quad DFS($G$, $v$, $T$)
		\EndFor
		\State \Return $count$
		\EndProcedure
	\end{algorithmic}
\end{algorithm}

\textbf{Oszacowanie złożoności:} Analogicznie do oszacowania DFS, co daje $O(n + m)$. Przyjmując, że
graf wejśćiowy ma co najmniej $n$ krawędzi otrzymamy złożoność  $O(m)$.

\subsection{Sprawdzanie dwudzielności grafu (2.5)}
\label{exc:bipart}
\paragraph{Treść.} Skonstruuj algorytm sprawdzający, 
czy zadany graf o $m$ krawędziach jest dwudzielny, w czasie $O(m)$.

\paragraph{Rozwiązanie.} Zauważmy najpierw, że 
graf $G$ jest dwudzielny wtedy i tylko wtedy, gdy istnieje $2$-kolorowanie
poprawne grafu $G$. \daggerfootnote{Dowód ($\Rightarrow$): kolorujemy 
wierzchołki obu zbiorów dwudzielności na przeciwne kolory. Dowód
($\Leftarrow$): kolorujemy poprawnym kolorowaniem, niebieskie
wierzchołki wyznaczają jeden zbiór, a czerwone drugi zbiór dwudzielności.} 

Aby rozwiązać ten problem, stosujemy kolorowanie 
zachłanne z przeszukiwaniem BFS. 
Wierzchołek, od którego zaczynamy przeszukiwanie, kolorujemy bez straty
ogólności na kolor niebieski, natomiast jego wszystkich sąsiadów na kolor czerwony.
Dzięki temu podczas przeszukiwania BFS będziemy zawsze badali pokolorowane 
wierzchołki. W przypadku kiedy badany wierzchołek jest w kolorze niebieskim, jego 
sąsiadów, którzy jeszcze nie zostali pokolorowani, pokolorujemy na kolor czerwony (analogicznie
w przypadku natrafienia na wierzchołek w kolorze czerwonym).

Zauważmy, że w przypadku, kiedy badamy wierzchołek w kolorze niebieskim oraz wśród jego sąsiadów
istnieje wierzchołek w kolorze niebieskim, kolorowanie jest niepoprawne, a zatem z udowodnionej 
równoważności graf nie jest dwudzielny.

Zakończenie powyższego przebiegu BFS oznacza, że pokolorowaliśmy cały zbiór wierzchołków na
dwa kolory. Zatem z udowodnionej równoważności
wierzchołki w kolorze niebieskim stanowią jedną klasę dwudzielności, a czerwone -- drugą.

Pozostaje pokazać, że istnieje 2-kolorowanie grafu $G$ $\Leftrightarrow$ opisane kolorowanie się powiedzie. 

Dowód ($\Leftarrow$) dla powyższego stwierdzenia jest oczywisty. Udowodnijmy, że ($\Rightarrow$). Pokolorujmy graf $G$
poprawnym 2-kolorowaniem i weźmy dowolny wierzchołek $x \in V(G)$. Zauważmy, że sąsiedzi
wierzchołka $x$ mają kolor przeciwny, bo kolorowanie jest poprawne. Teraz, jeśli dla każdego
sąsiada $x$, tj. wierzchołka ze zbioru $N(x)$, rozważymy te wierzchołki, których minimalna odległość do wierzchołka 
$x$ wynosi 2 (mówimy, że te wierzchołki są na poziomie 2), to będą one pokolorowane na ten sam 
kolor co $x$. Analizując kolejne poziomy
będziemy dostawili naprzemienne kolory. Zauważmy, że dokładnie w taki sposób nasz algorytm
będzie kolorował wierzchołki. Twierdzenie jest zatem prawdziwe.

\paragraph{Oszacowanie złożoności.} Przebiega analogicznie do zadania nr 4.

\subsection{Sortowanie topologiczne (2.6)}
\label{zad:tsort}
\paragraph{Treść.} Skonstruuj algorytm znajdujący 
sortowanie topologiczne acyklicznego grafu skierowanego 
o $m$ krawędziach w czasie $O(m)$. 
Przez sortowanie topologiczne rozumiemy uporządkowanie 
wierzchołków w taki sposób, że dla
każdej krawędzi $uv$, wierzchołek $u$ znajduje się przed 
wierzchołkiem $v$

\textbf{Rozwiązanie:} Rozwiązanie polega na zastosowaniu przeszukiwania DFS, które
odbywa się w funkcji \textsc{RSort}.

Za każdym razem na początek listy $L$ wrzucane są wierzchołki, 
które mają wszystkich odwiedzonych sąsiadów (lub nie mają sąsiadów),
co zapewnia poprawność algorytmu.

\begin{algorithm}[H]
	\caption{Sortowanie topologiczne wierzchołków drzewa}
	\begin{algorithmic}[1]
		\Procedure{TopologicalSort}{$G$: graf, $v$: wierzchołek startowy}
		\State $V' \gets V(G) \cup \{x\}$
		\State $E' \gets E(G) \cup \{xv : v \in V(G)\}$
		\State $G' \gets (V', E')$
		\State $T \gets$ tablica typu \textit{bool} $|V(G)|$ elementach zainicjowaną wartością 
		\textit{false}
		\State $L \gets$ pusta lista
		\State RSort($G'$, $x$, $L$)
		\State $L$.PopFront() \Comment Pierwszym elementem jest $x \not \in V(G)$
		\State \Return $L$
		\EndProcedure
		
		\Procedure{RSort}{$G$: graf, $v$: wierzchołek startowy, $L$: lista, $T$: tablica typu \textit{bool}}
		\While{$x \in N(v)$}
		\If {$T[x] = $ \textit{false} }
		\State $T[x] \gets$ \textit{true}
		\State TSort($G$, $x$, $L$, $T$)
		\EndIf
		\EndWhile
		\State $L$.PushFront(x)
		\EndProcedure
	\end{algorithmic}
	\label{alg:Zadanie26}
\end{algorithm}

Zauważmy, że dodanie do drzewa wierzchołka $x$, które odbywa się w linijkach 2, 3 i 4, nie może spowodować
cykli, bo w $G'$ nie istnieje łuk wchodzący do tego wierzchołka $x$. To w połączeniu z faktem, 
że $G$ jest drzewem oznacza, że $G'$ również jest drzewem.

\subsection{Szukanie cyklu Eulera (2.7)}
\paragraph{Treść.} Skonstruuj algorytm znajdujący cykl Eulera 
w zadanym grafie o $m$ krawędziach w czasie $O(m)$.

% \paragraph{Rozwiązanie.} :(
