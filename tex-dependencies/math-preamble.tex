\documentclass[12pt, a4paper]{article}
\usepackage{bookmark}
\usepackage{fullpage} % uniform 1.5 margins
\usepackage[polish]{babel}
\usepackage{polski}
\let\babellll\lll % babel and math shenanigans :/
\let\lll\relax
\usepackage[top=2cm, bottom=4.5cm, left=2.5cm, right=2.5cm]{geometry}  % page customization
\usepackage[T1]{fontenc} % use 8-bit font encoding
\usepackage{amsmath,amsthm,amsfonts,amssymb} % lovely American Mathematical Society packages
\let\mathlll\lll
\let\lll\babellll
\frenchspacing
\usepackage[makeroom]{cancel} % usefull cancelling (slash) command
\usepackage[dvipsnames,table]{xcolor} % colors!!!
\usepackage{float} % stop figures going to the top of the page u stupido by [H] placement
\usepackage{mathtools} % more cool math features
\usepackage{enumitem} % cool enumeration additions
\usepackage{fancyhdr} % fancy headers and footers
\usepackage{mathrsfs} % \mathscr command for cool font
\usepackage{graphicx} % enhanced support for graphics
\usepackage{hyperref} % extensive support for hypertext
\usepackage{theoremref} % refer to theorems using /ref
\usepackage{systeme} % french lib for systems of eq-ns
\usepackage{empheq} % emphasise eq-ns
\usepackage{multicol} % cool columns!
\usepackage{nicematrix} % nice matrices
\usepackage{tikz}  % plots!!
\usepackage{makecell} % cool table cell manipulation
\usepackage{calc} % arithmetic operations on internal counters (e.g. skipping enumeration)
\usepackage{stackengine} % stack engine; stack things on top of things!
\usepackage{diagbox} % table head with diagonal lines
\usepackage{yfonts} % support for old German fonts: gothic, schwabacher, fraktur and baroque
\usepackage{wrapfig} % wrap text around figures
\usepackage{mathdots} % more math dots!
\usepackage[bottom]{footmisc} % some footnote things
\usepackage[justification=centering]{caption} % customize captions in floats
\usepackage{subcaption} % subcaptions for subfigures
\usepackage[all]{nowidow} % avoid widows!
\usepackage{extarrows} % more arrows!
\usepackage{multirow}
\usepackage{amsmath} % fonts for algorithms
\usepackage{algorithm}
\usepackage{algorithmicx} % algorithms, pseudocode
\usepackage[noend]{algpseudocode} % algorithms, pseudocode

\usepackage{listings}
\usepackage{xcolor}
\lstset { %
    language=C++,
    backgroundcolor=\color{black!5}, % set backgroundcolor
    basicstyle=\footnotesize,% basic font setting
}
\input{insbox} % Floating box tex macro

\makeatletter
\def\BState{\State\hskip-\ALG@thistlm}
\makeatother

\renewcommand\useanchorwidth{T} % https://tex.stackexchange.com/questions/205145/flip-the-underbrace
\renewcommand\theadalign{bc} % table head align - bottom center
\renewcommand\theadfont{\bfseries} % table head font
\renewcommand\theadgape{\Gape[4pt]} % some table spacings ig
\renewcommand\cellgape{\Gape[4pt]}

\widowpenalty10000
\clubpenalty10000

% hypertext config
\hypersetup{
	colorlinks=true,
	citecolor=black,
	filecolor=black,
	linkcolor=black,
	urlcolor=black
}

% rename Rysunek to Rys.
\addto\captionspolish{%
	\renewcommand{\figurename}{Rys.}
}

% make space of width of one text and fill with other
\newcommand{\textover}[3][l]{%
	% #1 is the alignment, default l
	% #2 is the text to be printed
	% #3 is the text for setting the width
	\ensuremath{\makebox[\widthof{#3}][#1]{#2}}%
}

% matrix of type: ... ... ... | ...
\newenvironment{amatrix}[1]{%
	\begin{bNiceArray}{*{#1}{c}|c}
	}{%
	\end{bNiceArray}
}

% use boondox font on \mathds command
\DeclareMathAlphabet{\mathds}{U}{BOONDOX-ds}{m}{n}

\setlength{\parindent}{0.0in}
\setlength{\parskip}{0.05in}

% theorems with * are not numbered
\newtheorem{axiom}{Aksjomat}[section]
\newtheorem{theorem}{Twierdzenie}[section]
\newtheorem{defi}{Definicja}[section]
\newtheorem{corollary}{Wniosek}[theorem]
\newtheorem*{corollary*}{Wniosek}
\newtheorem{lemma}{Lemat}[section]
\newtheorem*{lemma*}{Lemat}
\newtheorem{excercise}{Zadanie}[section]
\newtheorem*{excercise*}{Zadanie}
\newtheorem*{example}{Przykład}
\newtheorem*{note}{Uwaga}
\newtheorem*{przp}{Przypomnienie}
\newtheorem{prop}{Własność}[section]
\newtheorem{fact}{Fakt}[section]
\newtheorem{algo}{Algorytm}[section]

% things needed for my basic template to work
\newcommand\person{Michał Okurowski}
\newcommand\depart{Wydział MiNI PW}

% slightly higher fbox? dont remember
\newcommand*\widefbox[1]{\fbox{\hspace{2em}#1\hspace{2em}}}

% footer and header setup
\pagestyle{fancyplain}
\headheight 35pt
\lhead{\person\\\depart}
%\chead{\textbf{\Large \dtype}\\\leftmark}
%\rhead{\course \\ \today}
\lfoot{}
\cfoot{}
\rfoot{\small\thepage}
\headsep 1.5em

% custom command for putting things on top of = sign
\newcommand{\myeq}[2][=]{\mathrel{\overset{#2}{#1}}}

% math sets defined
\newcommand\N{\ensuremath{\mathbb{N}}}
\newcommand\R{\ensuremath{\mathbb{R}}}
\newcommand\Z{\ensuremath{\mathbb{Z}}}
\newcommand\Q{\ensuremath{\mathbb{Q}}}
\newcommand\C{\ensuremath{\mathbb{C}}}
\renewcommand\d{\ensuremath{\mathrm{d}}}

% force \limits after \lim, old lim available as \svlim
\let\svlim\lim\def\lim{\svlim\limits}
% remove those ugly ass characters ew
\let\Re\relax 
\let\Im\relax
% \arccot is something wrong
\let\arccot\relax

% define my own operators
\DeclareMathOperator{\Re}{Re}
\DeclareMathOperator{\Im}{Im}
\DeclareMathOperator{\Arg}{Arg}
\DeclareMathOperator{\diag}{diag}
\DeclareMathOperator{\rz}{rz}
\DeclareMathOperator{\tr}{tr}
\DeclareMathOperator{\arccot}{arccot}

\usepackage{cutwin} % cut windows in paragraphs
\usepackage{pgfplots} % even better plots!!!
\pgfplotsset{compat=1.17}
\usetikzlibrary{matrix,calc,patterns,angles,quotes}
\usetikzlibrary{arrows.meta, intersections, fillbetween,babel}

% you know \cancelto? yea its always cancelling upwards, there is no \bcancelto, so here it is
\makeatletter
% #1, #2 offset of label   #6 extra width to clear arrowhead
% #3, #4 vector direction  #7 superscript label style
% #5 vector width          #8 superscript label
\def\cantox@vector#1#2#3#4#5#6#7#8{%
	\dimen@.5\p@
	\setbox\z@\vbox{\boxmaxdepth.5\p@
		\hbox{\kern-1.2\p@\kern#1\dimen@$#7{#8}\m@th$}}%
	\ifx\canto@fil\hidewidth  \wd\z@\z@ \else \kern-#6\unitlength \fi
	\ooalign{%
		\canto@fil$\m@th \CancelColor
		\vcenter{\hbox{\dimen@#6\unitlength \kern\dimen@
				\multiply\dimen@#4\divide\dimen@#3 \vrule\@depth\dimen@\@width\z@
				\vector(#3,-#4){#5}%
		}}_{\raise-#2\dimen@\copy\z@\kern-\scriptspace}$%
		\canto@fil \cr
		\hfil \box\@tempboxa \kern\wd\z@ \hfil \cr}}
\def\bcancelto#1#2{\let\canto@vector\cantox@vector\cancelto{#1}{#2}}
\makeatother

% some cute colors
\definecolor{highlight}{HTML}{FFFEEC}
\definecolor{highlight2}{HTML}{FAFFEC}

% \highlight[<color>]{<stuff>}
\newcommand{\highlight}[2][highlight]{\mathchoice%
	{\colorbox{#1}{$\displaystyle#2$}}%
	{\colorbox{#1}{$\textstyle#2$}}%
	{\colorbox{#1}{$\scriptstyle#2$}}%
	{\colorbox{#1}{$\scriptscriptstyle#2$}}}%

% Default plot config
\pgfplotsset{every axis/.append style ={
	width=1\linewidth,
	axis equal image,
	axis x line=center,
	axis y line=center,
	xlabel style={at={(current axis.right of origin)},anchor=north},
	ylabel style={at={(current axis.above origin)},anchor=east},
	trig format plots=rad,
	unbounded coords=jump,
	xlabel={$x$},
	ylabel={$y$},
}}

% some cute open and closed points
\tikzset{
	circ/.style={circle,fill=white,draw,inner sep=1.5pt,outer sep=0pt, fill opacity=1},
	poin/.style={circ,fill=#1,draw=#1,},
	poin/.default=black,
}

\newcounter{daggerfootnote}
\newcommand*{\daggerfootnote}[1]{%
	\setcounter{daggerfootnote}{\value{footnote}}%
	\renewcommand*{\thefootnote}{\fnsymbol{footnote}}%
	\footnote[2]{#1}%
	\setcounter{footnote}{\value{daggerfootnote}}%
	\renewcommand*{\thefootnote}{\arabic{footnote}}%
}

\let\originalleft\left
\let\originalright\right
\renewcommand{\left}{\mathopen{}\mathclose\bgroup\originalleft}
\renewcommand{\right}{\aftergroup\egroup\originalright}

\usepackage{lmodern}