\subsection{Algorytm Bellmana-Forda z odtwarzaniem ścieżki (3.1)}
\paragraph{Treść.}Zmodyfikuj algorytm Bellmana-Forda tak, 
aby znajdował najkrótszą ścieżkę między zadanymi dwoma
wierzchołkami.

\paragraph{Rozwiązanie.}

\begin{algorithm}[H]
	\caption{Algorytm Bellmana-Forda z odtwarzaniem ścieżki}
	\begin{algorithmic}[1]
		\Procedure{BellmanFord}{$(G, w)$: graf ważony, $s$: wierzchołek startowy, $t$: wierzchołek docelowy}

		\State odległość = tablica liczb, rozmiaru $V[G]$
		\State poprzednik = tablica wierzchołków, rozmiaru $V[G]$
		\For{$v \in V(G)$}
		\State odległość$[v]\gets\infty$
		\State poprzednik$[v]\gets-1$
		\EndFor
		\State odległość$[s]\gets0$
		\For{$i=1,2,\dots,n-1$}
		\For{$uv \in E(G)$}
		\If{odległość$[v] >$ odległość$[u]$ + $w(uv)$}
		\State odległość$[v]\gets$ odległość$[u] + w(uv)$ 
		\State poprzednik$[v]\gets u$
		\EndIf
		\EndFor
		\EndFor
		\State ścieżka $\gets$ pusta lista wierzchołków
		\State ściezka.PushFront($t$)
		\While{$t \gets \text{poprzednik}[t] \not = -1$}
		\State ścieżka.PushFront($t$)
		\EndWhile
		\State \Return ścieżka
		\EndProcedure
	\end{algorithmic}
	\label{Zadanie31}
\end{algorithm}
\subsection{Szukanie ujemnego cyklu (3.2)}
\paragraph{Treść.}Zaprojektuj algorytm, który znajdzie w 
ważonym grafie skierowanym ujemny cykl, jeśli taki istnieje.
Wskazówka: zmodyfikuj algorytm Bellmana-Forda.

\paragraph{Rozwiązanie.}
Rozwiązanie polega na sprawdzeniu, czy wykonanie
jeszcze jednej nadmiarowej iteracji 
w algorytmie Bellmana-Forda %robi co??
, oraz sprawdzenie
czy pewien wierzchołek został poprawiony. % ale co z tym sprawdzeniem?

Jeżeli poniższy warunek jest spełniony
\[\exists_{v\in V(G)}\quad \text{odległość}_n[v] < \text{odległość}_{n-1}[v],\]
to w grafie $G$ musi istnieć ujemny cykl. Dzieje się 
tak, ponieważ zawsze istnieje co najmniej jedna krawędź 
w cyklu ujemnym $C$, która spełnia warunek 
\[\text{odległość}[v] > \text{odległość}[u] + w(uv),\]
zatem będzie wtedy istniał wierzchołek, który zostanie 
poprawiony.

Jako że graf wejściowy może być niespójny,
musimy dodać dodatkowy wierzchołek $s$, który
będzie połączony z każdym innym wierzchołkiem
przy pomocy krawędzi skierowanej od wierzchołka $s$. Jako
że żadna krawędź nie wchodzi do wierzchołka $s$, możemy mieć pewność,
że żadne nowe cykle nie powstaną. Zatem
w szczególności nie powstaną nowe
ujemne cykle.

\begin{algorithm}[H]
	\caption{Znajdowanie ujemnego cyklu}
	\begin{algorithmic}[1]
		\Procedure{BellmanFord}{($G, w$): graf ważony}
		\State $G'$ $\gets$ graf $G$ z dodanym wierzchołkiem $s$
		oraz krawędziami skierowanymi od $s$ do każdego innego
		wierzchołka o wadze $0$
		\State odległość $\gets$ tablica liczb, rozmiaru $V[G']$
		\For{$v \in V(G')$}
		\State odległość$[v]\gets\infty$
		\EndFor
		\State odległość$[s]\gets0$
		\For{$i=1,2,\dots,n-1$}
		\For{$uv \in E(G')$}
		\If{odległość$[v] >$ odległość$[u]$ + $w(uv)$}
		\State odległość$[v]\gets$ odległość$[u] + w(uv)$ 
		\EndIf
		\EndFor
		\EndFor
		\For{$uv \in E(G')$}
		\If{odległość$[v] >$ odległość$[u]$ + $w(uv)$}
		\State return \true
		\EndIf
		\EndFor
		\State \Return \false
		\EndProcedure
	\end{algorithmic}
	\label{Zadanie32}
\end{algorithm}

\subsection{Szukanie liczby spacerów o zadanej liczbie krawędzi (3.3)}
\paragraph{Treść.}
Zaprojektuj algorytm, który dla danego grafu prostego 
$G$, wierzchołków $u, v \in V(G)$ i liczby $k \in
\{1, 2, \ldots , n\}$ znajdzie liczbę spacerów 
od $u$ do $v$ o dokładnie $k$ krawędziach.

%\paragraph{Rozwiązanie.}% :(

\subsection{Wyznaczanie odległości w drzewie ważonym (3.4)}
\paragraph{Treść.}Zaprojektuj algorytm, który dla ważonego 
acyklicznego grafu skierowanego $G$ oraz wierzchołków $s, t \in
V(G)$ wyznaczy odległość od $s$ do $t$ w czasie $O(m)$. 

Wskazówka: wykorzystaj sortowanie topologiczne (zadanie \ref{zad:tsort}).

\paragraph{Rozwiązanie.}
Rozwiązanie polega na zastosowaniu algorytmu zbliżonego do
algorytmu Dijkstry. Zamiast 
kolejki priorytetowej $Q$ wykorzystamy przygotowaną przez algorytm 
\ref{alg:Zadanie26} listę wierzchołków $L$ posortowanych topologicznie.
Nie ma potrzeby zmiany kolejności wierzchołków na liście $L$ w czasie 
wykonywania algorytmu, tak jak jest to robione w przypadku algorytmu
Dijkstry poprzez operację \textsc{DecreaseKey} wywoływanej na kolejce $Q$.

Jako że startowy wierzchołek nie musi być pierwszym wierzchołkiem na liście $L$,
musimy usunąć wszystkie wierzchołki z listy, które znajdują się przed $s$.

\begin{algorithm}[H]
	\caption{Znajdowanie ujemnego cyklu}
	\begin{algorithmic}[1]
		\Procedure{FindDistanceInTree}{$T$: drzewi, $\omega$: ważenie, $s$: wierzchołek startowy, $t$: wierzchołek docelowy}
		\State \textit{odległość} $\gets$ tablica liczb, rozmiaru $V[G]$
		\For{$v \in V(G)$}
		\State \textit{odległość}$[v] \gets \infty$
		\EndFor
		\State \textit{odległość}$[s] \gets 0$
		\State $L \gets$ \textsc{TopologicalSort}($T$) 
		\State Usuń z $L$ wszystkie wierzchołki przed $s$
		\While{$L$ nie jest pusta}
		\State $u \gets L$.DetachHead()
		\For{$v \in N(u)$} 
		\If{\textit{odległość}$[u] + \omega(uv) <$ \textit{odległość}$[v]$}
		\State \textit{odległość}$[v] \gets $ {odległość}$[u] + \omega(uv)$

		\EndIf
		\EndFor
		\EndWhile
		\State \Return \textit{odległość}$[t]$
		\EndProcedure
	\end{algorithmic}
	\label{Zadanie34}
\end{algorithm}

\subsection{Naiwna wersja algorytmu A\texorpdfstring{$^*$}{TEXT} (3.5)}
\paragraph{Treść.}Znajdź przykład, dla 
którego algorytm A$^*$ nie zadziała zgodnie z oczekiwaniami, 
w którym spełnione
są wszystkie założenia 
poza 
,,dla każdej krawędzi $uv$ zachodzi $h(u) \leq h(v) + w(uv)$''.

%\paragraph{Rozwiązanie.}% :/

\subsection{Poprawność algorytmu A\texorpdfstring{$^*$}{TEXT} (3.6)}
\paragraph{Treść.}Udowodnij poprawność algorytmu 
A$^*$ i wskaż, w których miejscach wykorzystywane są jego założenia.
Wskazówka: zainspiruj się dowodem poprawności algorytmu Dijkstry.

\paragraph{Rozwiązanie.}Dowód twierdzenia \ref{aStar_proof}.

\subsection{Poprawność algorytmu Floyda-Warshalla (3.7)}
\paragraph{Treść.}Udowodnij poprawność algorytmu Floyda-Warshalla. 
Wskazówka: niech 
$d^{(m)}(i, j)$ będzie długością
najkrótszej ścieżki od $i$ do $j$, której wewnętrzne wierzchołki należą
do zbioru $\{0, 1, 2, . . . , m - 1\}$; 
udowodnij, że po $l$
iteracjach zewnętrznej pętli dla każdych $i, j \in V (G)$ zachodzi 
odległość$[i, j] \leq d^{(l)}(i, j)$.

\paragraph{Rozwiązanie.}Dowód twierdzenia \ref{floydWarshall_proof}.
