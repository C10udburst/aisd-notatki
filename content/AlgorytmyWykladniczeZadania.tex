\subsection{Zadanie 1 -- Najmniejszy zbiór dominujący}
\paragraph{Treść:}
\textit{Zbiór dominujący} w grafie $G$ to podzbiór wierzchołków 
$D \subset V(G)$ taki, że każdy wierzchołek grafu $G$
należy do $D$ lub ma przynajmniej jednego sąsiada w $D$.

Zaprojektuj jak najwydajniejszy algorytm, który znajdzie rozmiar najmniejszego zbioru dominującego w zadanym
grafie o $n$ wierzchołkach i maksymalnym stopniu nie większym niż 3. Oszacuj złożoność tego algorytmu.

\paragraph{Rozwiązanie:}
Aby rozwiązać to zadanie zastosujemy algorytm z nawrotami.

Wprowadźmy oznaczenia, które zostaną wykorzystane w algorytmie.
Niech $D_{\min}$ to najmniejszy zbiór dominujący jaki udało 
nam się dotychczas znaleźć.
Niech $D$ to aktualny stan zbioru dominującego, oraz
niech $X$ to zbiór wierzchołków, które na pewno nie znajdą się 
w zbiorze dominującym $D$ na następnych poziomach rekurencji.
Warunkiem stopu będzie sytuacja, w której $|D| + |X| = |V(G)|$.

Możemy wprowadzić optymalizację polegająca na tym, że 
przestaniemy badać daną gałąź wykonań rekurencyjnych,
w przypadku, kiedy $|D| < |D_{\min}|$. Optymalizacji tej nie będziemy
stosowali w przypadku kiedy $D_{\min}$ nie zostało jeszcze znalezione,
w tym celu skorzystamy z faktu, że jeśli graf $G$, jest 
taki, że $V(G) \not = \emptyset$,
to każdy zbiór dominujący dla tego grafu nie jest pusty.

W poniższym algorytmie, przyjmujemy, że $D_{\min}$ to globalny zbiór
wierzchołków, który początkowo jest pusty.
\begin{algorithm}[H]
	\caption{Algorytm znajdowania najmniejszego zbioru dominującego}
	\begin{algorithmic}[1]
		\Procedure{DominatingSet}{$G$: graf, $D$: zbiór, $X$: zbiór}
		\If{$|D| > |D_{min}| \land |D_{min}| \not = \emptyset$}
		\State \Return
		\EndIf
		\State Niech $Y$ to zbiór wierzchołków niezdominowanych, tzn. takich, które nie mają żadnego sąsiada w zbiorze $D$ (żaden
		wierzchołek z $Y$ nie należy do $X$ ani $D$)		

		\If{$Y = \emptyset$}
		\If{Każdy wierzchołek z $X$ jest zdominowany}
		\State $D_{\min} = D$
		\EndIf
		\State \Return
		\EndIf
		\State Niech $u$ to dowolny wierzchołek z $Y$
		\State DominatingSet($G$, $D \cup \{u\}$, $X$, $D_{\min}$)
		\If{$|N(u)| \geq 1$}
		\State Niech $v_1 \in N(X)$
		\If{$v_1 \not \in D \land v_1 \not \in X$}
		\State DominatingSet($G$, $D \cup \{v_1\}$, $X \cup \{u\}$, $D_{\min}$)
		\EndIf
		\EndIf
		\If{$|N(u)| \geq 2$}
		\State Niech $v_2 \in N(X) \setminus \{v_1\}$
		\If{$v_2 \not \in D \land v_2 \not \in X$}
		\State DominatingSet($G$, $D \cup \{v_2\}$, $X \cup \{u, v_1\}$, $D_{\min}$)
		\EndIf
		\EndIf
		\If{$|N(u)| = 3$}
		\State Niech $v_3 \in N(X) \setminus \{v_1, v_2\}$
		\If{$v_3 \not \in D \land v_3 \not \in X$}
		\State DominatingSet($G$, $D \cup \{v_3\}$, $X \cup \{u, v_1, v_2\}$, $D_{\min}$)
		\EndIf
		\EndIf
		\EndProcedure		
	\end{algorithmic}
\end{algorithm}

Aby oszacować złożoność powyższego algorytmu skorzystamy z 
twierdzenia o rozwiązywaniu rekurencji. Przez $f(n)$
oznaczmy pesymistyczny czas działania funkcji \textsc{DominatingSet},
wtedy prawdą jest, że
\[f(n) \leq f(n-1) + f(n-1) + f(n-2) + f(n-3) + f(n-4) + O(n),\]
gdzie $O(n)$ to złożoność znajdowania zbioru $Y$. 
Szukamy pierwiastka poniższego równania
\[x^4 = x^3 + x^2 + x + 1,\]
po rozwiązaniu powyższego równania, otrzymamy największy
pierwiastek $x \approx 1.9276$. Zatem na mocy twierdzenia
o rozwiązywaniu rekurencji złożoność algorytmu wynosi $O(n1.9276^n) = O^*(1.9276^n)$.

\subsection{Zadanie 2 -- Problem 3-SAT}

\paragraph{Treść: } Niech $x$ będzie zmienną mogącą przyjmować wartość \textit{true} lub \textit{false}. Przez literał rozumiemy zmienną
$x$ lub jej zaprzeczenie $\lnot x$. Klauzulą nazywamy zbiór literałów -- na przykład $\{x, y, \lnot z\}$. Mówimy, że wartościowanie
zmiennych spełnia klauzulę, jeżeli przynajmniej jeden z jej literałów jest prawdziwy.
\\ W problemie 3-SAT mamy dane $n$ zmiennych $x_1, x_2, \ldots, x_n$ oraz zbiór klauzul, z których każda zawiera co najwyżej
3 literały. Pytamy, czy istnieje wartościowanie zmiennych, które spełnia każdą klauzulę.
Zaprojektuj jak najszybszy algorytm rozwiązujący problem 3-SAT i oszacuj jego złożoność.

\paragraph{Rozwiązanie: } Rozważmy algorytm naiwny (używający \textit{bruteforce}'a). Będziemy brać zmienną $x$, która jeszcze nie ma wartościowania i przypisywać jej wartość \true{} lub \false, a następnie rekurencyjnie wywoływać funkcję dla kolejnych zmiennych. Na każdym poziomie rekurencji liniowo sprawdzamy, czy wszystkie klauzule zostały spełnione lub któraś jest niespełniona. Opisany algorytm działa w czasie $O^*(2^n)$ (liczba wszystkich ciągów binarnych długości $n$).


Algorytm naiwny możemy ulepszyć, zauważając następujący fakt: do spełnienia klauzuli wystarczy, żeby co najmniej jeden z literałów miał wartość \textit{true}. Dodatkowo zmienne będziemy rozpatrywać w kolejności pojawiania się w jeszcze nierozstrzygniętych klauzulach.

Mając nierozstrzygniętą klauzulę $\{x, y, z\}$, w której żadna zmienna nie ma przypisanej wartości, będziemy rozpatrywać następujące wartościowania\daggerfootnote{Naturalnie przypadków będzie mniej, jeżeli klauzula ma mniej literałów lub któraś ze zmiennych ma już przypisaną wartość -- wtedy rozważamy odpowiednio jeden lub dwa pierwsze przypadki.}:

\begin{align*}
	\begin{cases}
		x = 1;
	\end{cases}
	\qquad
	\begin{cases}
		x = 0, \\
		y = 1;
	\end{cases}
	\qquad
	\begin{cases}
		x = 0, \\
		y = 0,\\
		z = 1.
	\end{cases}
\end{align*}

W ten sposób maksymalnie osiem możliwych przypadków wartościowania zmiennych w klauzuli $\{x, y, z\}$ ograniczyliśmy do trzech. 

\begin{algorithm}[H]
	\caption{Rozwiązanie problemu 3-SAT}
	\begin{algorithmic}[1]
		\Procedure{Solve3SAT}{$\mathfrak{K}$: zbiór klauzul nierozstrzygniętych}
		% \If {przy obecnym wartościowaniu któraś z klauzul z $\mathfrak{K}$ jest niespełnialna}
		% 	\State \Return \false
		% \EndIf
		\If {$\mathfrak{K} = \emptyset$}
		\State \Return \true
		\EndIf
		
		\State 
		\State \textit{Kasia} $\gets$ nierozstrzygnięta klauzula z $\mathfrak{K}$ 
		\If{\textit{Kasia} jest niespełnialna przy obecnym wartościowaniu}
		\State \Return \false
		\EndIf
		\State $x, y, z \gets $ literały \textit{Kasi} bez wartościowania
		\State
		\State $x \gets $ \true
		\State $\mathfrak{K}' \gets \mathfrak{K}$ bez \textit{Kasi} i klauzul spełnionych przez $x$
		\If {\textsc{Solve3SAT}($\mathfrak{K}'$)}
		\State \Return \true
		\EndIf
		\State
		\State $x \gets $ \false, $y \gets $ \true \Comment{Jeśli literałów było $\geq 2$}
		\State $\mathfrak{K}' \gets \mathfrak{K}$ bez \textit{Kasi} i klauzul spełnionych przez $y$
		\If {\textsc{Solve3SAT}($\mathfrak{K}'$)} 
		\State \Return \true
		\EndIf
		\State
		\State $x \gets $ \false, $y \gets $ \false, $z \gets $ \true \Comment{Jeśli literałów było 3}
		\State $\mathfrak{K}' \gets \mathfrak{K}$ bez \textit{Kasi} i klauzul spełnionych przez $z$
		\If {\textsc{Solve3SAT}($\mathfrak{K}'$)}
		\State \Return \true
		\EndIf
		\EndProcedure
	\end{algorithmic}
\end{algorithm}

Oznaczmy jako $f(n)$ pesymistyczny czas działania funkcji \textsc{Solve3SAT}, gdzie $n$ jest liczbą zmiennych bez przypisanej wartości. Możemy oszacować:

\[
f(n) \leq f(n-1) + f(n-2) + f(n-3) + O(n),
\]

gdzie $O(n)$ wynika z pesymistycznie liniowego czasu obliczania wartości $\mathfrak{K}'$. Pierwiastkiem równania
$x^3 = x^2 + x + 1$
jest $x \approx 1{,}8393$, zatem z twierdzenia o rozwiązywaniu rekurencji pesymistyczna złożoność algorytmu wynosi $O^*(1{,}8393^n)$.

\subsection{Zadanie 3 -- Kolorowanie grafu z list o rozmiarze nie większym niż 2}
\paragraph{Treść:} W problemie kolorowania grafu $G$ z list dla każdego wierzchołka 
$v \in V(G)$ mamy daną listę $L_v$ i
szukamy poprawnego kolorowania grafu $G$, w którym kolor każdego wierzchołka $v$ należy do $L_v$.
Zaprojektuj algorytm, który rozwiąże problem kolorowania grafu z list w czasie wielomianowym w przypadku, gdy
wszystkie listy mają rozmiar nie większy niż $2$.

\paragraph{Rozwiązanie:} 

Niech graf $G$ to graf wejściowy w naszym algorytmie, taki, że
$V(G) = \{v_1, v_2, \ldots, v_n\}$. 
Niech $\chi = \bigcup_{i = 1}^n L_{v_i}$.
Utwórzmy zbiór wierzchołków
\[V' = \{v_{ij} : j \in L_{v_i} \land i \in [n] \land j \in \chi\},\]

oraz zbiór krawędzi 
\[E' = \{v_{ij}v_{kl} : i = k \land i,k \in [n] \land j, l \in \chi\} \cup 
\]
\[\cup
\{v_{ij}v_{kl} : j = l \land v_iv_k \in E(G) \land i,k \in [n] \land j, l \in \chi\}.\]

Niech graf $G' = (V', E')$. 

Prostrzymi słowami, każdy wierzchołek z $V(G)$ rozbijamy na dwa wierzchołki odpowiadające kolorom z listy (lub jeden, jeśli na liście znajduje się tylko jeden kolor) i tworzymy pomiędzy 
nimi krawędź. Pozostałe krawędzie pomiędzy dwoma wierzchołkami
$u$ i $v$ z $G'$ tworzymy wtedy kiedy wierzchołki 
z których $u$ i $v$ zostały rozbite sąsiadują ze sobą
w grafie $G$ oraz $u$ i $v$ odpowiadają temu samemu kolorowi.

Wystarczy roztrzygnąć dwudzielność 
grafu $G'$, przy pomocy algorytmu, który zastosowaliśmy w zadaniu 5 z tematu ,,Przeszukiwanie grafów''. Jeśli graf $G'$ jest dwudzielny
to możemy pokolorować go listowo.

Dzieje się tak, dlatego, że w klasie dwudzielności nigdy
nie znajdą się dwa wierzchołki $u, v \in V'$ rozbite z tego samego wierzchołka, bo z definicji połączyliśmy je krawędzią.

Oznacza to, że dla każdego wierzchołka $x \in V(G)$ musi
istnieć dokładnie jeden wierzchołek z niego rozbity znajdujący się w takiej klasie dwudzielności. Jako, że w grafie $V(G')$
mogą sąsiadować tylko te wierzchołki, które odpowiadają temu samemu kolorowi, to żaden kolor w takiej klasie dwudzielności nie może się powtórzyć.

Utworzenie zbiorów jak i algorytm roztrzygania dwudzielności wymagają czasu wielomianowego, więc rozwiązanie spełnia wymagania czasowe.

