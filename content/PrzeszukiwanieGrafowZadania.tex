\subsection{Zadanie 1 -- Struktura danych reprezentująca graf skierowany}

\textbf{Treść:} Zaprojektuj strukturę danych reprezentującą graf skierowany o $n$ wierzchołkach i $m$ krawędziach, z
następującymi operacjami:
\begin{itemize}
	\item dodanie krawędzi,
	\item usunięcie krawędzi,
	\item sprawdzenie istnienia krawędzi,
	\item przejrzenie wszystkich krawędzi wychodzących z zadanego wierzchołka $v$,
\end{itemize}
która spełnia następujące wymagania:
\begin{itemize}
	\item[a)] Operacje na pojedynczej krawędzi w czasie $O(\log n)$, przejrzenie krawędzi wychodzących z $v$ w czasie $O(\deg^+(v))$\footnote{$\deg^+(v)$ oznacza liczbę krawędzi wychodzących z wierzchołka $v$.}, 
	złożoność pamięciowa $O(m)$.
	\item[b)] Dodawanie krawędzi w czasie $O(1)$, przejrzenie krawędzi wychodzących z $v$ w czasie $O(\deg^+(v))$, złożoność
	pamięciowa $O(m)$.
	\item[c)] Operacje na pojedynczej krawędzi w czasie $O(1)$, przejrzenie krawędzi wychodzących z $v$ w czasie $O(\deg^+(v))$.
	\item[d)] Operacje na pojedynczej krawędzi w oczekiwanym czasie $O(1)$, przejrzenie krawędzi wychodzących z $v$ w czasie
	$O(\deg^+(v))$, złożoność pamięciowa $O(m)$.
\end{itemize}

\textbf{Rozwiązanie}
\begin{itemize}
	\item[a)] Wystarczy zastosować reprezentację listową, w której 
	listy zastępujemy drzewami zrównoważonymi (CC, AVL).
	\item[b)] Lista nieuporządkowana dla każdego wierzchołka
	\item[c)] Zastosować reprezentację macierzową, która przechowuje
	referencję na node w liscie incydencji (mamy liste incydencji
	gdzie sa wartosci i tablice referencji)
	\item[d)] Tablica haszująca o $m$ elementach (haszowanie łańcuchowe)
\end{itemize}

\subsection{Zadanie 2 -- Nierekurencyjne przeszukiwanie w głąb (DFS)}
\textbf{Treść: } Zaproponuj nierekurencyjną implementację
przeszukiwania grafu w głąb.

\textbf{Rozwiązanie:}
Poniższy algorytm działa zarówno dla grafów skierowanych
jak i prostych.

\begin{algorithm}[H]
	\caption{Rozwiązanie zadania 2}\label{Zadanie22a}
	\begin{algorithmic}[1]
		\Procedure{DFSNR($G$ -- graf, $v$ -- wierzchołek startowy)}{}
		\State Zainicjalizuj stos (\textit{last in, first out}) $S$
		\State Zainicjalizuj tablicę booli $T$ o $|V(G)|$ elementach wartością 
		\textit{false}
		\State $S$.Push(v)
		\State $T[v] \gets$ \textit{false}
		\While {$S$.IsEmpty() = \textit{false}}
		\State $x \gets$ $S$.Pop()
		\While {$y \in N(x)$}
		\If {$T[y]$ = \textit{false}}
		\State $S$.Push(y)
		\State $T[y] \gets$ \textit{true} 
		\EndIf
		\EndWhile
		\EndWhile
		\EndProcedure
	\end{algorithmic}
\end{algorithm}

\textbf{Wersja rekurencyjna:} Zakładamy, że dysponujemy globalną
tablicą booli $T$ o $|V(G)|$ elementach zainicjowaną wartością 
\textit{false}.

\begin{algorithm}[H]
	\caption{Algorytm przeszukiwania DFS -- wersja rekurencyjna}\label{Zadanie22b}
	\begin{algorithmic}[1]
		\Procedure{DFS($G$ -- graf, $v$ -- wierzchołek startowy)}{}
		\While{$x \in N(v)$}
		\If {$T[x] = $ \textit{false} }
		\State $T[x] \gets$ \textit{true}
		\State DFS($G$, $x$)
		\EndIf
		\EndWhile
		\EndProcedure
	\end{algorithmic}
\end{algorithm}

\subsection{Zadanie 3 -- Implementacja przeszukiwania wszerz (BFS)}

\textbf{Treść: } Zaproponuj implementację
przeszukiwania grafu wszerz.

\textbf{Rozwiązanie:}
Poniższy algorytm działa zarówno dla grafów skierowanych
jak i prostych.

\begin{algorithm}[H]
	\caption{Rozwiązanie zadania 2}\label{Zadanie23}
	\begin{algorithmic}[1]
		\Procedure{BFS($G$ -- graf, $v$ -- wierzchołek startowy)}{}
		\State Zainicjalizuj kolejkę (\textit{first in, first out}) $Q$
		\State Zainicjalizuj tablicę bool'i $T$ o $|V(G)|$ elementach wartością 
		\textit{false}
		\State $Q$.PushBack(v)
		\State $T[v] \gets$ \textit{false}
		\While {$Q$.IsEmpty() $=$ \textit{false}}
		\State $x \gets$ $Q$.PopFront()
		\While {$y \in N(x)$}
		\If {$T[y] =$  \textit{false}}
		\State $Q$.PushBack(y)
		\State $T[y] \gets$ \textit{true} 
		\EndIf
		\EndWhile
		\EndWhile
		\EndProcedure
	\end{algorithmic}
\end{algorithm}

\subsection{Zadanie 4 -- Szukanie liczby składowych spójności grafu}
\textbf{Treść: } konstruuj algorytm znajdujący liczbę składowych 
spójności zadanego grafu w czasie $O(m)$

\textbf{Rozwiązanie: } Stosujemy BFS lub DFS z dowolnego 
wierzchołka i zapisujemy odwiedzone wierzchołki (tablica bool).
Jeżeli po wyszukiwaniu istnieje nieodwiedzony wierzchołek to 
wykonujemy na nim BFS lub DFS ponownie. Powtarzamy aż
wszystkie wierzchołki będą odwiedzone. 

Liczba wywołań BFS lub DFS to liczba składowych w grafie $G$.

\subsection{Zadanie 5 -- Sprawdzanie dwudzielności grafu}
\textbf{Treść: } Skonstruuj algorytm sprawdzający, 
czy zadany graf o $m$ krawędziach jest dwudzielny, w czasie $O(m)$.

\textbf{Rozwiązanie: } Zauważmy, najpierw, że 
graf $G$ jest dwudzielny $\Leftrightarrow$ istnieje $2$-kolorowanie
poprawne grafu $G$. (Dowód ($\Rightarrow$): kolorujemy 
wierzchołki obu zbiorów dwudzielności na przeciwne kolory. Dowód 
($\Leftarrow$): kolorujemy poprawnym kolorowaniem, niebieskie
wierzchołki wyznaczają jeden zbiór, a czerwone drugi zbiór dwudzielności). 

Aby rozwiązać ten problem, stosujemy kolorowanie 
zachłanne z przeszukiwaniem BFS (dowód że to działa -\textbf{ to do}). 
Podczas wyszukiwania wierzchołek, od którego zaczynamy 
wyszukiwanie, kolorujemy 
bez straty ogólności na czerwono, natomiast jego
wszystkich sąsiadów na niebiesko.

Powtarzamy wywołanie BFS tak długo, aż nie będzie 
wierzchołków niepokolorowanych.

\subsection{Zadanie 6 -- Sortowanie topologiczne}
\textbf{Treść: } Skonstruuj algorytm znajdujący 
sortowanie topologiczne acyklicznego grafu skierowanego 
o $m$ krawędziach w czasie $O(m)$. 
Przez sortowanie topologiczne rozumiemy uporządkowanie 
wierzchołków w taki sposób, że dla
każdej krawędzi $uv$, wierzchołek $u$ znajduje się przed 
wierzchołkiem $v$

\textbf{Rozwiązanie:} Zakładamy, że dysponujemy globalną
tablicą booli $T$ o $|V(G)|$ elementach zainicjowaną wartością 
\textit{false}.

Ponadto zakładamy, że mamy globalną listę $L$, w której 
będą znajdować się posortowane wierzchołki.

\begin{algorithm}[H]
	\caption{Rozwiąznie zadania 5}
	\begin{algorithmic}[1]
		\Procedure{TSort($G$ -- graf, $v$ -- wierzchołek startowy)}{}
		\While{$x \in N(v)$}
		\If {$T[x] = $ \textit{false} }
		\State $T[x] \gets$ \textit{true}
		\State TSort($G$, $x$)
		\EndIf
		\EndWhile
		\State $L$.PushFront(x)
		\EndProcedure
	\end{algorithmic}
	\label{Zadanie26}
\end{algorithm}

\subsection{Zadanie 7 -- Szukanie cyklu Eulera}
\textbf{Treść: } Skontruuj algorytm znajdujący cykl Eulera 
w zadanym grafie o $m$ krawędziach w czasie $O(m)$.
\textbf{Rozwiązanie: }