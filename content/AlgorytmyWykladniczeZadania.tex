\subsection{Zadanie 3 -- Kolorowanie grafu z list o rozmiarze nie większym niż 2}
\paragraph{Treść:} W problemie kolorowania grafu $G$ z list dla każdego wierzchołka 
$v \in V(G)$ mamy daną listę $L_v$ i
szukamy poprawnego kolorowania grafu $G$, w którym kolor każdego wierzchołka $v$ należy do $L_v$.
Zaprojektuj algorytm, który rozwiąże problem kolorowania grafu z list w czasie wielomianowym w przypadku, gdy
wszystkie listy mają rozmiar nie większy niż $2$.

\paragraph{Rozwiązanie:} 

Niech graf $G$ to graf wejściowy w naszym algorytmie, taki, że
$V(G) = \{v_1, v_2, \ldots, v_n\}$. 
Niech $\chi = \bigcup_{i = 1}^n L_{v_i}$.
Utwórzmy zbiór wierzchołków
\[V' = \{v_{ij} : j \in L_{v_i} \land i \in [n] \land j \in \chi\},\]

oraz zbiór krawędzi 
\[E' = \{v_{ij}v_{kl} : i = k \land i,k \in [n] \land j, l \in \chi\} \cup 
\]
\[\cup
\{v_{ij}v_{kl} : j = l \land v_iv_k \in E(G) \land i,k \in [n] \land j, l \in \chi\}.\]

Niech graf $G' = (V', E')$. 

Prostrzymi słowami, każdy wierzchołek z $V(G)$ rozbijamy na dwa wierzchołki odpowiadające kolorom z listy (lub jeden, jeśli na liście znajduje się tylko jeden kolor) i tworzymy pomiędzy 
nimi krawędź. Pozostałe krawędzie pomiędzy dwoma wierzchołkami
$u$ i $v$ z $G'$ tworzymy wtedy kiedy wierzchołki 
z których $u$ i $v$ zostały rozbite sąsiadują ze sobą
w grafie $G$ oraz $u$ i $v$ odpowiadają temu samemu kolorowi.

Wystarczy roztrzygnąć dwudzielność 
grafu $G'$, przy pomocy algorytmu, który zastosowaliśmy w zadaniu 5 z tematu ,,Przeszukiwanie grafów''. Jeśli graf $G'$ jest dwudzielny
to możemy pokolorować go listowo.

Dzieje się tak, dlatego, że w klasie dwudzielności nigdy
nie znajdą się dwa wierzchołki $u, v \in V'$ rozbite z tego samego wierzchołka, bo z definicji połączyliśmy je krawędzią.

Oznacza to, że dla każdego wierzchołka $x \in V(G)$ musi
istnieć dokładnie jeden wierzchołek z niego rozbity znajdujący się w takiej klasie dwudzielności. Jako, że w grafie $V(G')$
mogą sąsiadować tylko te wierzchołki, które odpowiadają temu samemu kolorowi, to żaden kolor w takiej klasie dwudzielności nie może się powtórzyć.

Utworzenie zbiorów jak i algorytm roztrzygania dwudzielności wymagają czasu wielomianowego, więc rozwiązanie spełnia wymagania czasowe.