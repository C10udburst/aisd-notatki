\subsection{Zadanie 1 -- Szukanie podzbioru o zadanej sumie elementów}

\paragraph{Treść: } Zaprojektuj algorytm, który dla zadanego zbioru 
liczb naturalnych $S$ i liczby naturalnej $N$ rozstrzygnie,
czy $S$ zawiera podzbiór o sumie równej $N$ w czasie $O(N |S|)$.
Wskazówka: wyznacz wszystkie możliwe sumy podzbiorów zbioru $S$.

\paragraph{Rozwiązanie: } Niech $S = \{s_0, s_1, \dots, s_{|S| - 1}\}$ będzie zbiorem wejściowym oraz 
niech $N$ będzie liczbą, dla której sprawdzimy czy istnieją elementy, które się do niej sumują.

Niech $T$ będzie tablicą dwuwymiarową o wymiarach $|S|$ na $N + 1$, przyjmująca wartości 
\textit{True} lub \textit{False} w każdej komórce tablicy $T$.
Komórka tablicy $T[i, j]$ odpowiada na pytanie, czy da się otrzymać 
sumę równą $i$ z $j$-elementowego podzbioru $\{s_1, s_2, \dots, s_j\} \subseteq S$.

Algorytm polega na uzupełnianiu kolejnych wierszy tablicy.
Najpierw rozważamy problem oparty na zbiorze $\{s_1\}$. 
Nastepnie aby otrzymać rozwiązanie dla $\{s_1, s_2\}$, wystarczy
od każdej komórki przechowującej $T$ przemieścić się o $s_2$ komórek w prawo
i zmienić wartość na $T$ i tak do momentu uzupełnienia tablicy $T$.

Po uzupełnieniu $T$ odpowiedzią na zadane pytanie jest $T[k, N]$.

Rozważmy przykład: $S = \{1, 3, 5, 10\}$ (tzn. $s_0=1$, $s_1=3$, $s_2=5$, $s_3=10$), 
$N = 9$. Tablica prezentuje się wtedy jak w tabeli \ref{tab:problemnptrudny}.

\begin{table}[H]
	\centering
	\begin{tabular}{|l|l|l|l|l|l|l|l|l|l|l|}
		\hline
		& \multicolumn{1}{l|}{\textbf{0}} & \multicolumn{1}{l|}{\textbf{1}} & \multicolumn{1}{l|}{\textbf{2}} & \multicolumn{1}{l|}{\textbf{3}} & \multicolumn{1}{l|}{\textbf{4}} & \multicolumn{1}{l|}{\textbf{5}} & \multicolumn{1}{l|}{\textbf{6}} & \multicolumn{1}{l|}{\textbf{7}} & \multicolumn{1}{l|}{\textbf{8}} & \multicolumn{1}{l|}{\textbf{9}} \\ \hline
		\textbf{1 $\{1\}$} & \color{ForestGreen}T & \color{ForestGreen}T & F & F & F & F & F & F & F & F \\ \cline{1-1}
		\textbf{2 $\{1, 3\}$} & \color{ForestGreen}T & \color{ForestGreen}T & F & \color{ForestGreen}T & \color{ForestGreen}T & F & F & F & F & F \\ \cline{1-1}
		\textbf{3 $\{1, 3, 5\}$} & \color{ForestGreen}T & \color{ForestGreen}T & F & \color{ForestGreen}T & \color{ForestGreen}T & \color{ForestGreen}T & \color{ForestGreen}T & F & \color{ForestGreen}T & \color{ForestGreen}T \\ \cline{1-1}
		\textbf{4 $\{1, 3, 5, 10\}$} & \color{ForestGreen}T & \color{ForestGreen}T & F & \color{ForestGreen}T & \color{ForestGreen}T & \color{ForestGreen}T & \color{ForestGreen}T & F & \color{ForestGreen}T & \color{ForestGreen}T  \\ \cline{1-1}
		\hline
	\end{tabular}
	\caption{Elementy $s_1, s_2 \dots, s_k \in S$ nie muszą być posortowane.}
	\label{tab:problemnptrudny}
\end{table}

\begin{algorithm}[H]
	\caption{Szukanie podzbioru o zadanej sumie elementów}\label{Zadanie11}
	\begin{algorithmic}[1]
		\Procedure{SubsetSum}{$S = \{s_0, s_1, \dots, s_{|S| - 1}\} \subseteq \mathbb{N}$: zbiór, $N \in \mathbb{N}$: szukana suma}
		\State Utwórz tablicę dwuwymiarową $T$ o $|S|$ wierszach i $N + 1$ kolumnach 
		\State Wypełnij tablicę $T$ wartością \textit{False}
		\State $T[0, 0] \gets $ \textit{True}
		\State $T[0, s_1] \gets $ \textit{True}
		\State $i \gets 1$
		\For{$i = 1, 2 \dots |S| - 1$}
		\For{$j = 0, 1, 2 \dots N$}
		\If{$T[i - 1, j] = $ \textit{True}}
		\State $T[i, j] \gets \textit{True}$
		\If{$j + s_j \leq N$}
		\State $T[i, j + s_j] \gets$ \textit{True}
		\EndIf
		\EndIf
		\EndFor 
		\EndFor
		\State \Return $T[k, N]$
		\EndProcedure 
	\end{algorithmic}
\end{algorithm}

\subsection{Zadanie 2 -- Szukanie najdłuższego wspólnego podciągu}
\textbf{Treść:} Zaprojektuj algorytm, który znajdzie długość najdłuższego wspólnego podciągu dwoch zadanych ciągów
o nie więcej niż $n$ wyrazach w czasie $O(n^2)$. Przykładowo, najdłuższym wspólnym podciągiem 
ciągów „abcdef” i „eacgd” jest „acd”.

Wskazówka: zastosuj programowanie dynamiczne; wyznacz rozwiązanie dla każdej pary prefiksów zadanych napsów.

\textbf{Rozwiązanie:}
Oznaczmy ciągi wejściowe jako 
$A = a_1, a_2, \dots a_n$ oraz 
$B = b_1, b_2, \dots b_m$. Niech $C$ będzie największym możliwym
wspólnym podciągiem $A$ i $B$.  Zauważmy, że jeśli znamy rozwiązanie
nastepujących podproblemów:
\begin{enumerate}
	\item największy możliwy wspólny podciąg $C_1$ dla ciągów $A_1=a_1, a_2, \dots a_n$ oraz 
	$B_1=b_1, b_2, \dots b_{m-1}$,
	\item największy możliwy wspólny podciąg $C_2$ dla ciągów $A_2 =a_1, a_2, \dots a_{n-1}$ oraz 
	$B_2=b_1, b_2, \dots b_{m}$,
	\item największy możliwy wspólny podciąg $C_3$ dla ciągów $A_3=a_1, a_2, \dots a_{n-1}$ oraz 
	$B_3=b_1, b_2, \dots b_{m-1}$,
\end{enumerate}
to jesteśmy w stanie znaleźć $C$
w następujący sposób: jeżeli $a_n = b_m$, to rozwiązaniem 
jest podciąg $C_3$ z dodatkowym elementem $a_n$, w przeciwnym przypadku
rozwiązaniem musi być większy z ciągów $C_1$ oraz $C_2$ ($\ast$). 

Aby zaimplementować to rozumowanie, zastosujemy tablicę dwuwymiarową $T$ 
(indeksujemy od 0)
o $n+1$ wierszach oraz $m+1$ kolumnach. Jako że w zadaniu mowa jest 
tylko o długości szukanego podciągu $C$, każda z komórek tablicy $T$ 
o indeksach $i$ oraz $j$ ($i \in \{0,1, \dots, n\}$, 
$j \in \{0,1, \dots, m\}$) będzie 
przechowywać długość szukanego podciągu dla ciągów 
$a_1, a_2, \dots, a_i$ oraz $b_1, b_2, \dots, b_j$. 

Dla przykładu 
jeśli $A = $ „abcdef” i  $B = $ „eacgd”, to tablica 
wygląda tak jak tabela \ref{tab_zad12}.


\begin{table}[H]
	\center
	\begin{tabular}{|l|lllllll|}
		\hline
		& \multicolumn{1}{l|}{\textbf{""}} & \multicolumn{1}{l|}{\textbf{a}} & \multicolumn{1}{l|}{\textbf{b}} & \multicolumn{1}{l|}{\textbf{c}} & \multicolumn{1}{l|}{\textbf{d}} & \multicolumn{1}{l|}{\textbf{e}} & \multicolumn{1}{l|}{\textbf{f}} \\ \hline
		\textbf{""} & 0 & 0 & 0 & 0 & 0 & 0 & 0 \\ \cline{1-1}
		\textbf{e}  & 0 & 0 & 0 & 0 & 0 & 1 & 1 \\ \cline{1-1}
		\textbf{a}  & 0 & 1 & 1 & 1 & 1 & 1 & 1 \\ \cline{1-1}
		\textbf{c}  & 0 & 1 & 1 & 2 & 2 & 2 & 2 \\ \cline{1-1}
		\textbf{g}  & 0 & 1 & 1 & 2 & 2 & 2 & 2 \\ \cline{1-1}
		\textbf{d}  & 0 & 2 & 2 & 2 & 3 & 3 & 3 \\ \cline{1-1}
		\hline
	\end{tabular}
	\caption{}
	\label{tab_zad12}
\end{table}

Na początku zerową kolumnę oraz zerowy wiersz tablicy $T$
wypełniamy zerami. W każdym kroku algorytmu, wypełniając $T[i, j]$
($i \in \{0,1, \dots, n\}$, $j \in \{0,1, \dots, m\}$)
będziemy analizowali $T[i-1, j]$, $T[i, j-1]$ oraz $T[i-1,j-1]$ wg ($\ast$).


\begin{algorithm}[H]
	\caption{Szukanie najdłuższego wspólnego podciągu}\label{Zadanie12}
	\begin{algorithmic}[1]
		\Procedure{SubsetSum}{$A = a_1, a_2, \dots, a_n$, $B = b_1, b_2, \dots, b_m$}
		\State Utwórz tablicę dwuwymiarową $T$ o $n+1$ wierszach i $m+1$ kolumnach.
		\State Wypełnij zerową kolumnę i zerowy wiersz tablicy $T$ zerami
		\State $i \gets 1$
		\While{$i \leq n$}
		\State $j \gets 1$
		\While{$j \leq m$}
		\If {$a_i = b_j$}
		\State $T[i, j] = T[i-1,j-1] + 1$
		\Else
		\State $T[i, j] = \max\{T[i-1,j], T[i,j-1]\}$
		\EndIf
		\State $j \pp$
		\EndWhile
		\State $i \pp$
		\EndWhile
		\State \Return $T[n, m]$
		\EndProcedure 
	\end{algorithmic}
\end{algorithm}


\subsection{Zadanie 3 -- Problem łamania tekstu}
\textbf{Treść:} W problemie łamania tekstu dany jest tekst składający się z $n$
słów o długościach $d_1, d_2, \ldots , d_n$ oraz
szerokość wiersza $s$. Szukamy rozmieszczenia słów w wierszach w taki 
sposób, żeby każde słowo mieściło się w całości
w wierszu, w żadnym wierszu nie bylo więcej niż $s$ znaków 
oraz suma kwadratów liczb pozostałych wolnych miejsc we
wszystkich wierszach była jak najmniejsza.
Zaprojektuj jak najszybszy algorytm rozwiązujący problem łamania tekstu.

\textbf{Rozwiązanie:} Niech wejściem algorytmu będzie ciąg  długości słów $d=d_1,d_2,\dotsc,d_n$ oraz szerokość linii $s$.

Niech $C$ będzie tablicą dwuwymiarową o wymiarach $n$ na $n$, gdzie w komórce $C[i,j]$ znajdować będzie się koszt umieszczenia słów od $i$ do $j$ w jednej linii, gdzie jako koszt rozumiemy kwadrat liczby pozostałych miejsc. Dla każdej pary $(i,j)$, $i\leq j$ należy więc wyznaczyć ilość dodatkowych spacji. Mamy więc przypadki: \begin{itemize}
	\item słowa od $i$ do $j$ nie mieszczą się w linii: $C[i,j]=\infty$,
	\item słowa od $i$ do $j$ mieszczą się, ale $j=n$ (jest to ostatnia linia), więc nie ma kosztu: $C[i,j]=0$,
	\item w przeciwnym wypadku: $C[i,j]=\textrm{ex}^2$, gdzie $\textrm{ex}$ to liczba dodatkowych spacji w linii.
\end{itemize}

Przykładowo dla $d=(3,2,2,5)$ i $s=6$ tablica $C$ wygląda jak tabela \ref{tab_zad13_C}.
\begin{table}[H]
	\centering
	\def\arraystretch{1.25}
	\begin{tabular}{|c|cccc|}
		\hline \diagbox{$i$}{$j$}
		& \multicolumn{1}{c|}{$d_1$} & \multicolumn{1}{c|}{$d_2$} & \multicolumn{1}{c|}{$d_3$} & \multicolumn{1}{c|}{$d_4$} \\ \hline
		$d_1$ & 9 & 0  & $\infty$ & $\infty$ \\ \cline{1-1}
		$d_2$ &   & 16 &        1 & $\infty$ \\ \cline{1-1}
		$d_3$ &   &    &       16 & $\infty$ \\ \cline{1-1}
		$d_4$ &   &    &          &        0 \\ \cline{1-1}
		\hline
	\end{tabular}
	\caption{}
	\label{tab_zad13_C}
\end{table}

Mając wyliczone wszystkie koszty możemy znaleźć najmniejszy całkowity koszt. Niech $c$ będzie tablicą długości $n+1$, gdzie wartość $c[j]$ informuje, jaki jest najmniejszy koszt rozmieszczenia słów od $1$ do $j$. Dodatkowo niech $c[0]=0$. Zauważmy, że z definicji $c[j]$ można skonstruować następującą zależność rekurencyjną: \[
	c[j]=\begin{cases}
		0, &j=0\\
		\min\limits_{1 \leq i \leq j}\left(c[i-1]+C[i,j]\right), &j>0
	\end{cases}
\]
Rekurencja ta polega na wykorzystaniu wyliczonych już minimalnych kosztów rozmieszczenia słów i dodaniu nowej linii ze słowami od $i$ do $j$. Koszt takiego nowego łamania linii to właśnie $c[i-1] + C[i,j]$ (minimalny koszt użycia słów od $1$ do $i-1$ plus koszt nowej linii).

Jednocześnie niech $p$ będzie tablicą długości $n+1$, gdzie w $p[j]$ wpisany będzie indeks $i$, dla którego osiągnięto minimum w trakcie wyliczania $c[j]$.

Dla wcześniejszego przykładu $d=(3,2,2,5)$ i $s=6$ tablice $c$ i $p$ wyglądają jak tabela \ref{tab_zad13_c}.
\begin{table}[H]
	\centering
	\def\arraystretch{1.25}
	\begin{tabular}{|c|ccccc|}
		\hline & \multicolumn{1}{c|}{0} & \multicolumn{1}{c|}{$d_1$} & \multicolumn{1}{c|}{$d_2$} & \multicolumn{1}{c|}{$d_3$} & \multicolumn{1}{c|}{$d_4$} \\ \hline
		$c$ & 0 & 9 & 0 & 10 & 10 \\ \cline{1-1}
		$p$ & 0 & 1 & 1 & 2 & 4 \\ \cline{1-1}
		\hline
	\end{tabular}
	\caption{}
	\label{tab_zad13_c}
\end{table}

Zauważmy, że $p[n]$ zawiera informację, od którego słowa zaczyna się linia kończąca się słowem $n$. Na podstawie tablicy $p$ można teraz odtworzyć poszczególne linie: ostatnia linia zawiera słowa od $p[n]$ do $n$, przedostatnia linia zawiera słowa od $p[p[n]-1]$ do $p[n]-1$ itd.

W algorytmie panuje umowa, że $C$ jest indeksowana od 1, a $c,p$ od 0.
\begin{algorithm}[H]
	\caption{Problem łamania tekstu}\label{Zadanie13}
	\begin{algorithmic}[1]
		\Procedure{BreakLines}{$d = (d_1, d_2, \dotsc, d_n)$: ciąg długości słów, $s$: szerokość linii}
			\State Utwórz tablicę dwuwymiarową $C$ o $n$ wierszach i $n$ kolumnach.
			\State Utwórz tablice $c,p$ o $n+1$ kolumnach.
			\For{$i\in\{1,2,\dotsc,n\}$}
				\State $\textrm{ex}\gets s+1$\Comment{$+1$, bo pierwsze słowo nie ma spacji}
				\For{$j\in\{i,i+1,\dotsc,n\}$}
					\State $\textrm{ex}\gets \textrm{ex} - d_j - 1$
					\If{$\textrm{ex}<0$}
						\State $C[i,j]\gets\infty$
					\ElsIf{$j=n$}
						\State $C[i,j]\gets0$
					\Else
						\State $C[i,j]\gets\textrm{ex}^2$
					\EndIf
				\EndFor
			\EndFor
			\State $c[0]\gets0$,\quad$p[0]\gets0$
			\For{$j\in\{1,2,\dotsc,n\}$}
				\State $c[j]\gets\infty$
				\For{$i\in{1,2,\dotsc,j}$}
					\If{$c[i-1]+C[i,j]<c[j]$}
						\State $c[j]\gets c[i-1]+C[i,j]$,\qquad$p[j]\gets i$
					\EndIf
				\EndFor
			\EndFor
			\State \Return $p$
		\EndProcedure
	\end{algorithmic}
\end{algorithm}

\subsection{Zadanie 4 -- Szukanie odległości edycyjnej napisów}
\textbf{Treść:} Przez odległość edycyjną napisów 
$x$ i $y$ rozumiemy najmniejszą liczbę operacji wstawienia lub usunięcia
pojedynczego znaku, które pozwalają przekształcić napis $x$ w napis $y$.
Zaprojektuj algorytm, który znajdzie odległość 
edycyjną dwóch zadanych napisów.


\textbf{Rozwiązanie:}
Warto wspomnieć, że dodanie jeszcze jednej operacji --
zamiany znaków z dwóch różnych słów powoduje powstanie metryki
którą nazywamy odległością Levenshteina.

Przeanalizujmy najpierw naiwne podejście rekurencyjne: badamy 
oba napisy od tyłu, oznaczmy $m$ jako długość napisu $x$
oraz $n$ jako długość napisu $y$. Rozważmy dwa przypadki: 

\begin{itemize}
	\item[1.] Jeśli znaki są takie same to wywołujemy rekurencyjnie
	nasz algorytm dla napisu $x$ skróconego do $m - 1$ oraz 
	napisu $y$ skróconego do $n - 1$.
	\item[2.] Jeśli znaki są różne to potrzebujemy dwóch wywołań rekurencyjnych
	\begin{itemize}
		\item wywołanie dla $m - 1$ oraz $n$, które interpretujemy jako operację usunięcia znaku $x[m]$,
		\item wywołanie dla $m$ oraz $n - 1$, które
		interpretujemy jako operację wstawienia znaku $y[m]$ do napisu $x$,
		po znaku $x[m]$.
	\end{itemize} 
\end{itemize}

W algorytmie \ref{Zadanie14-naive}, przyjmujemy konwencję numerowania napisów od 1.
\begin{algorithm}[H]
	\caption{Odległość edycyjna -- naiwny algorytm rekurencyjny}
	\begin{algorithmic}[1]
		\Procedure{NaiveEditDistance}{$x$: napis, $m$: długość $x$, $y$: napis, $n$: długość $y$}
		\If{$m=0$}
		\Comment $x$ jest puste, więc możemy jedynie dodać wszystkie znaki z napisu $y$ do napisu $x$
		\State \Return $n$ 
		\EndIf
		\If{$n=0$}
		\Comment $y$ jest puste, więc możemy jedynie usunąć wszystkie znaki z napisu $x$
		\State \Return $m$ 
		\EndIf
		\If{$x[m] = y[n]$}
		\State \Return \textsc{NaiveEditDistance}($x$, $m - 1$, $y$, $n - 1$)
		\Else
		\State \textit{DeleteCost} $\gets$ \textsc{NaiveEditDistance}($x$, $m - 1$, $y$, $n$)
		\State \textit{InsertCost} $\gets$ \textsc{NaiveEditDistance}($x$, $m$, $y$, $n - 1$)
		\State \Return $1 + \min(\textit{DeleteCost, InsertCost})$ \Comment +1, bo musimy wliczyć koszt wykonania aktualnej operacji
		\EndIf
		\EndProcedure 
	\end{algorithmic}
	\label{Zadanie14-naive}
\end{algorithm}

Teraz zajmijmy się podejściem programowania dynamicznego.
Niech $T$ będzie tablicą dwuwymiarową o
$n + 1$ wierszach oraz $m + 1$ kolumnach
gdzie kolumny oznaczają kolejne litery napisu $x$ a
wiersze kolejne litery napisu $y$. Komórka
$T[i, j]$ ($i \in [0, n+1], j \in [0, m+1]$), będzie 
oznaczała odległość 
podnapisu $x$ składającego się z indeksów $1,2 \ldots j$ od 
podnapisu $y$ składającego się z indeksów $1,2 \ldots i$.

Przyjmijmy, że $i \in [1, n+1]$ i $ j \in [1, m+1]$ oraz 
zaobserwujmy następującą relację: 
\begin{itemize}
	\item jeśli $x[j]=y[i]$, to $T[i, j] = T[i - 1, j - 1]$,
	\item w przeciwnym przypadku $T[i, j] = 1 + \min(T[i - 1, j], T[i, j - 1])$, gdzie
	$+1$ wynika, z potrzeby wykonania operacji dodawania lub usuwania znaku
	(w zależności od tego, czy $T[i - 1, j] > T[i, j - 1]$).
\end{itemize}

Przykład: rozważmy dwa napisy $x =$ ,,piesek'' oraz $y =$ ,,kotek''.

\begin{table}[H]
	\center
	\begin{tabular}{|c|lllllll|}
		\hline \diagbox{$y$}{$x$}
		& \multicolumn{1}{l|}{\textbf{""}} & \multicolumn{1}{l|}{\textbf{p}} & \multicolumn{1}{l|}{\textbf{i}} & \multicolumn{1}{l|}{\textbf{e}} & \multicolumn{1}{l|}{\textbf{s}} & \multicolumn{1}{l|}{\textbf{e}} & \multicolumn{1}{l|}{\textbf{k}} \\ \hline
		\textbf{""} & 0 & 1 & 2 & 3 & 4 & 5 & 6 \\ \cline{1-1}
		\textbf{k}  & 1 & 2 & 3 & 4 & 5 & 6 & 5 \\ \cline{1-1}
		\textbf{o}  & 2 & 3 & 4 & 5 & 6 & 7 & 6 \\ \cline{1-1}
		\textbf{t}  & 3 & 4 & 5 & 6 & 7 & 8 & 7 \\ \cline{1-1}
		\textbf{e}  & 4 & 5 & 6 & 5 & 6 & 7 & 8 \\ \cline{1-1}
		\textbf{k}  & 5 & 6 & 7 & 6 & 7 & 8 & 7 \\ \cline{1-1}
		\hline
	\end{tabular}
	\caption{}
	\label{tab_zad14}
\end{table}

Na podstawie tabeli \ref{tab_zad14}, wiemy, że szukana odległość wynosi 7.

\begin{algorithm}[H]
	\caption{Rozwiązanie zadania 1.4}
	\begin{algorithmic}[1]
		\Procedure{EditDistance($x$, $m$, $y$, $n$)}{}
		\State Utwórz tablicę $T$ o $n + 1$ wierszach i $m + 1$ kolumnach
		\For{$i=0,1\ldots n$}
		\State $T[i, 0] = i$
		\EndFor		
		\For{$j=0,1\ldots m$}
		\State $T[0, j] = j$
		\EndFor
		\For{$i=1,2\ldots n$}
		\For{$j=1,2\ldots m$}
		\If{$x[j] = y[i]$}
		\State $T[i, j] = T[i - 1, j - 1]$
		\Else
		\State $T[i, j] = 1 + \min(T[i - 1, j], T[i, j - 1])$
		\EndIf
		\EndFor
		\EndFor
		\State \Return $T[n, m]$
		\EndProcedure 
	\end{algorithmic}
	\label{Zadanie14}
\end{algorithm}

