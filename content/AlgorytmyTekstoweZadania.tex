\subsection{Zadanie 1 -- najdłuższy prefiks}
\paragraph{Treść:}
Dany jest wzorzec $x$ oraz napis $y$. Zaprojektuj algorytm, który znajdzie najdłuższy prefiks $x$ zawarty w napisie $y$ w czasie $O(|x|+|y|)$.

\paragraph{Rozwiązanie:}
W zadaniu tym można wykorzystać modyfikację algorytmu KMP (\ref{KMP}), która zapamiętuje pozycję i długość najdłuższego znalezionego prefiksu.

\begin{algorithm}[H]
	\caption{Algorytm \emph{LongestMatchingPrefix}}
	\begin{algorithmic}[1]
		\Procedure{LongestMatchingPrefix}{$tekst$: napis, $wzorzec$: napis}
		\State $P \gets \textsc{ComputeP}(wzorzec)$
		\State $n \gets |tekst|$
		\State $m \gets |wzorzec|$
		\State $j \gets 0$
		\State $maxLen\gets-1$
		\State $maxPos\gets-1$ 
		\For{$i = 0;\ i \leq n - m;\ i \pleq \operatorname{max}(j - P[j], 1)$}
			\State {$j = P[j]$}
			\While {$j < m \land y[i+j]=x[j]$}
				\State $j\pp$
			\EndWhile
			\If{$j=m$}
				\State\Return $(maxPos, maxLen)$
			\ElsIf{$j>maxLen$}
				\State $maxLen\gets j$
				\State $maxPos\gets i$
			\EndIf
		\EndFor
		\State\Return $(maxPos, maxLen)$
		\EndProcedure		
	\end{algorithmic}
\end{algorithm}

\subsection{Zadanie 2 -- szukanie przynajmniej połowy wzorca}

\subsection{Zadanie 4 -- dopasowanie wzorca ze znakami nieznaczącymi typu '\texttt{?}'}
\label{zadanie8.4}
\paragraph{Treść:}
Problem dopasowania wzorca ze znakami nieznaczącymi różni się od problemu wyszukiwania wzorca w tekście tylko tym, że wzorzec może zawierać specjalny znak '\texttt{?}', który pasuje do każdego znaku.

Zaprojektuj algorytm rozwiązujący ten problem w czasie $O(nk)$, gdzie $n$ jest długością tekstu, a $k$ -- liczbą znaków nieznaczących we wzorcu.
\paragraph{Rozwiązanie:}
Dzielimy wzorzec wg. znaku '\texttt{?}' na $k+1$ podwzorców (być może niektóre są puste, bo kilka '\texttt{?}' stoi obok siebie, ale nie szkodzi). Tworzymy tablicę $n\times(k+1)$, gdzie na pozycji $(i,j)$ znajduje się informacja, czy na $i$-tej pozycji tekstu znajduje się $j$-ty podwzorzec. Korzystając z KMP, wyszukujemy każdy podwzorzec w tekście i zapisujemy w tablicy odpowiednią informację o jego obecności. Dla pustych podwzorców wypełniamy cały wiersz informacją o jego znalezieniu, żeby zapewnić sobie poprawne działanie dalszej części algorytmu.

Następnie przeglądamy wygenerowaną tablicę w celu znalezienia wystąpienia całego wzorca. Po znalezieniu wystąpienia pierwszego podwzorca, tj. na pozycji $(i_0, 0)$, przesuwamy się o jeden wiersz w dół i długość podwzorca plus jeden w prawo, to znaczy w miejsce, gdzie spodziewamy się dopasowania kolejnego podwzorca. Robimy tak, aż do znalezienia ostatniego podwzorca, gdyż wtedy dopasowaliśmy cały wzorzec. Jeśli którykolwiek podwzorzec po drodze nie został znaleziony, to przerywamy operację i przechodzimy do kolejnego wystąpienia pierwszego podwzorca, tj. $(i_1,0)$ i powtarzamy szukanie.\daggerfootnote{Możemy usprawnić ten algorytm zauważając następujący niezmiennik: kolejne podwzorce będą znajdować się w tekście dalej, niż poprzednie. Co za tym idzie, możemy wywołać KMP dopiero od pierwszego dopasowania poprzedniego podwzorca.}

Załóżmy, że szukamy wzorca ,,\texttt{AAB?XYZ}'' w tekście ,,\texttt{XYZAABAABCXYZA}''.
\begin{table}[H]

\centering
\begin{NiceTabular}{c|cccccccccccccc}[cell-space-limits=1.5mm]
tekst&X&Y&Z&A&A&B&A&A&B&C&X&Y&Z&A\\\hline
podwzorzec $0$\RowStyle{\color{ForestGreen}}&&&&A&A&B&A&A&B&&&&&\\
podwzorzec $1$\RowStyle{\color{ForestGreen}}&X&Y&Z&&&&&\color{red}X&&&X&Y&Z&\\
\CodeAfter
  \tikz \draw[ForestGreen] (1-|8) |- (3-|12) |- (4-|15) -- (1-|15); 
  \tikz \draw[red,dashed] (1-|5) |- (3-|9) |- (4-|10); 
\end{NiceTabular}
\caption{Przebieg algorytmu dla przytoczonego wyżej wzorca i tekstu. Najpierw wyszukiwane są położenia podwzorca ,,\texttt{AAB}'' (pozycje 3 i 6), a następnie podwzorca ,,\texttt{XYZ}'' (pozycje 0 i 10). Dla dopasowania podwzorca ,,\texttt{AAB}'' na pozycji 3, brakuje dopasowania wzorca ,,\texttt{XYZ}'' na pozycji $3+3+1=7$, zatem szukamy innego dopasowania podwzorca ,,\texttt{AAB}''. Kolejne takie dopasowanie znajduje się na pozycji 6, a na pozycji $6+3+1=10$ znajduje się dopasowanie podwzorca ,,\texttt{XYZ}'', zatem dopasowaliśmy cały wzorzec.}
\end{table}

Algorytm ten wywołuje KMP $k+1$ razy, zatem jego złożoność to $O(nk)$.

\subsection{Zadanie 5 -- dopasowanie wzorca ze znakami nieznaczącymi typu '\texttt{*}'}
\paragraph{Treść:}
Rozwiąż zadanie \ref{zadanie8.4} w wersji ogólniejszej, ze znakiem '\texttt{*}' we wzorcu, który pasuje do każdego ciągu znaków o długości nie większej niż $d$, dla zadanej stałej $d$.
