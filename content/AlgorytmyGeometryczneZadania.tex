\subsection{Zadanie 1 -- Sprawdzenie czy wielokąty wypukłe z użyciem zamiatania} 
\paragraph{Treść:} Zaprojektuj algorytm, który w 
czasie $O(n)$ sprawdzi, czy dwa zadane wielokąty wypukłe o nie więcej
niż $n$ wierzchołkach się przecinają. 

\paragraph{Rozwiązanie:}
Stosujemy metodę zamiatania. Zdefiniujmy \textit{zdarzenie},
jako typ danych, który przechowuje wierzchołek $p$, dwa
odcinki $s_1$, $s_2$, których końcem jest $p$ oraz informację 
do którego wielokątu należy ten wierzchołek.

Przyjmujemy, że na wejściu oba wielokąty będą
reprezentowane przez listę zdarzeń posortowanych
względem współrzędnej x wierzchołka $p$. 

Stosujemy zmodyfikowaną wersję algorytmu \ref{HasIntersectingSegments1}, który za 
pomocą metody zamiatania stwierdza, czy w podanym
zbiorze odcinków, istnieją dwa, które się przecinają.  

Aby zescalać dwie wejściowe listy reprezentujące wierzchołki pierwszego
i drugiego wielokątu możemy zastosować liniową metodę scalania
znaną z algorytmu MergeSort. Taka uporządkowana lista
będzie stanowiła X-strukturę.

Jako Y-strukturę możemy przyjąć dowolną strukturę danych, która
zapewni możliwość wykonania wymaganych operacji. Ze względu 
na to, że w Y-strukturze nie może znaleźć się więcej niż 
4 odcinki jednocześnie (bo każdy wierzchołek jest jednocześnie
końcem jednej krawędzi i początkiem drugiej oraz mamy 
2 wielokąty), możemy przyjąć, że operacje na Y-strukutrze wykonują się w czasie $O(1)$.

Jako, że sytuację w której jeden wielokąt znajduje się wewnątrz drugiego
wielokątu, również traktujemy jako przecięcie wielokątów, musimy nanieść modyfikację
na algorytm \ref{HasIntersectingSegments1}. Modyfikacja polega na tym, by 
sprawdzić, czy występuje sytuacja, w której dodany odcinek $s$
jednego wielokąta znajduje się nad odcinkiem $s_1$ oraz pod odcinkeim $s_2$,
które należą do drugiego wielokąta. Z wypukłości wielokątów wiemy, że 
taka sytuacja wystąpi tylko wtedy gdy wielokąty na siebie nachodzą, lub
jeden jest w drugim.
 
\begin{algorithm}[H]
	\caption{Sprawdzenie czy wielokąty się przecinają}
	\begin{algorithmic}[1]
		\Procedure{HasIntersectingSegments}{$S$: zbiór odcinków spełniający założenia}
		\State $X \gets \text{Merge}(L_1, L_2)$ 
		\State $Y \gets \emptyset$
		\While{$X \not = \emptyset$}
		\State \textit{zdarzenie} = $X.$DetachHead()
		\State $s_1, s_2 \gets \text{odcinki odpowiadające \textit{zdarzeniu}}$
		\State $p \gets \text{punkt odpowiadający \textit{zdarzeniu}}$
		\If{$p$ jest początkiem odcinika $s$}
		\State Dodaj $s$ do $Y$-Struktury
		\If {$s$ należy do jednego wielokąta oraz $Y$.Above($s$) lub z $Y$.Below($s$)
		należą do drugiego}
		\State \Return true
		\EndIf
		\If{$s$ przecina się z $Y$.Above($s$) lub z $Y$.Below($s$)}
		\State \Return true
		\EndIf
		\Else
		\If{$Y$.Above($s$) przecina się z $Y$.Below($s$)}
		\State \Return true
		\EndIf
		\EndIf
		\EndWhile
		\State \Return false
		\EndProcedure
	\end{algorithmic}
	\label{algZadanie7_1}
\end{algorithm}


